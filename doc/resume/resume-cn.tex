% Created 2017-04-20 木 18:02
\documentclass[11pt,dvipdfmx,CJKbookmarks]{article}
\usepackage{CJKutf8}
\usepackage[utf8]{inputenc}
\usepackage[T1]{fontenc}
\usepackage{fixltx2e}
\usepackage{graphicx}
\usepackage{longtable}
\usepackage{float}
\usepackage{wrapfig}
\usepackage{rotating}
\usepackage[normalem]{ulem}
\usepackage{amsmath}
\usepackage{textcomp}
\usepackage{marvosym}
\usepackage{wasysym}
\usepackage{amssymb}
\usepackage{hyperref}
\tolerance=1000
\usepackage{minted}
\usepackage{mycv}
\author{Bao Haojun}
\date{\today}
\title{包昊军的个人简历(电子版)}
\hypersetup{
  pdfkeywords={},
  pdfsubject={},
  pdfcreator={Emacs 25.1.91.1 (Org mode 8.2.10)}}
\AtBeginDvi{\special{pdf:tounicode UTF8-UCS2}}
\begin{document}
\begin{CJK*}{UTF8}{simsun}

\maketitle



\section{工作经历}
\label{sec-1}
\subsubsection{2014\thinspace 年\thinspace 3\thinspace 月 - 现在}
\label{sec-1-0-1}
\begin{itemize}
\item \textbf{CM\thinspace 架构师},锤子科技,北京
\begin{itemize}
\item 搭建、改善锤子科技\thinspace CM(Configuration Management)系统
\item 搭建锤子科技\thinspace CI(Continuous Integration)系统
\item 搭建锤子科技\thinspace SmartBuilder(自助版本编译、发布)系统
\item 制定、改进工程师开发流程,编写开发流程文档
\item 结合个人的\thinspace \href{https://github.com/baohaojun/system-config}{system-config}\thinspace 开源项目,统一工程师开发环境

简化了开发环境配置,提升了工程师的效率

\item 对工程师工作中碰到的各种疑难杂症提供支持
\item 对工程师提供开发工具、编程语言等培训

拍摄了两小时长的\thinspace Linux\thinspace 系统工具使用技巧视频,在\thinspace \href{https://www.youtube.com/watch?v\%3Dqp2b3-Guej0}{YouTube}\thinspace 上

\item 对\thinspace CM\thinspace 团队其他同事的技术培训和支持
\end{itemize}

\item \textbf{工具架构师},锤子科技,北京

\begin{itemize}
\item 负责部分工厂生产测试工具、工程师开发工具的设计、实现等
\item 设计、实现\thinspace PC\thinspace 端操作手机的开源软件 \href{https://github.com/SmartisanTech/Wrench}{小扳手}

并拍摄了使用视频,在\thinspace \href{https://www.youtube.com/watch?v\%3Dre_bECYY0rM&app\%3Ddesktop}{YouTube}\thinspace 上
\item 调试并解决手机过热问题
\end{itemize}
\end{itemize}

\subsubsection{2013\thinspace 年\thinspace 9\thinspace 月 - 2014\thinspace 年\thinspace 3\thinspace 月}
\label{sec-1-0-2}
\begin{itemize}
\item \textbf{技术专家},技术质量部测试工具组,Alibaba,北京
\begin{itemize}
\item 负责\thinspace SMART(软件测量分析报告工具)安卓客户端开发
\end{itemize}
\end{itemize}

\subsubsection{2011\thinspace 年\thinspace 11\thinspace 月 - 2013\thinspace 年\thinspace 9\thinspace 月}
\label{sec-1-0-3}
\textbf{Staff Engineer},\href{http://marvell.com}{Marvell},北京

\begin{itemize}
\item 在\thinspace Tools\thinspace 团队担任架构师的职务,负责\thinspace Marvell\thinspace 手机整体解决方案中工厂生
产工具的设计与实现
\begin{itemize}
\item 手机端测试\thinspace case\thinspace 全部用\thinspace Bash\thinspace 脚本实现,使开发流程更为高效

\item 用\thinspace C/C++ 实现所有必要的硬件测试\thinspace case:LCD(frame buffer)、
Touch(input device)、Camera(V4L2)、Audio(ALSA、Marvell API)、
CP(AT command)、Sensors/Wifi/BlueTooth(Android HAL
programming)

\item 试产时到深圳工厂跟踪此工具在产线的实际使用情况,根据产线反馈对
其进行改进
\end{itemize}

\item 在\thinspace BSP\thinspace 团队负责与工具相关的功能模块的设计与实现,包括\thinspace Uboot,Kernel\thinspace 接口,
工厂分区等

\begin{itemize}
\item 在\thinspace Uboot\thinspace 里参考原有的\thinspace usbtty\thinspace 代码(for old CPU and USB framework),
从头重新\thinspace bring up\thinspace 了\thinspace usbtty\thinspace 以解决\thinspace Form Factor\thinspace 上串口调试不便的问题
\item 实现在\thinspace Linux\thinspace 下用\thinspace fastboot\thinspace 对\thinspace Marvell\thinspace 方案的整机烧写,解决了工程师只
能在\thinspace Windows\thinspace 机器上用\thinspace Marvell\thinspace 私有工具做整机烧写的问题,提升同事的
开发效率
\end{itemize}

\item 结合对开源社区的认识,对与\thinspace Marvell\thinspace 其他团队之间的合作提出流程上的改
良建议

\begin{itemize}
\item 更新\thinspace Marvell Android Makefile\thinspace 脚本,使其支持并行编译和\thinspace ccache\thinspace 的使
用。Kernel\thinspace 代码编译时间从\thinspace 10min\thinspace 缩减到\thinspace 2min
\end{itemize}
\end{itemize}

\subsubsection{2010\thinspace 年\thinspace 3\thinspace 月 - 2011\thinspace 年\thinspace 10\thinspace 月}
\label{sec-1-0-4}

\textbf{Staff Engineer},RayzerLink/Letou

\begin{itemize}
\item 负责基于\thinspace Nvidia Tegra2\thinspace 芯片的平板电脑底层软件开发,主要包括\thinspace Linux
Kernel bring-up,驱动(Touch、LCD、Sensors),Hal(硬件抽象层)的
开发等工作

\item 辅导\thinspace BSP\thinspace 新同事底层系统开发工作
\end{itemize}

\subsubsection{2008\thinspace 年\thinspace 11\thinspace 月 - 2010\thinspace 年\thinspace 3\thinspace 月}
\label{sec-1-0-5}

\textbf{Senior Engineer},\href{http://www.borqs.com}{播思通讯}

\begin{itemize}
\item 在\thinspace Tools\thinspace 组工作

\item 设计并实现手机开发、测试、生产以及售后等各个环节需要用到的一系列
工具
\end{itemize}


\subsubsection{2005\thinspace 年\thinspace 9\thinspace 月 - 2008\thinspace 年\thinspace 9\thinspace 月}
\label{sec-1-0-6}

\textbf{Software Engineer},\href{http://motorola.com}{摩托罗拉}, MD/GSG

\begin{itemize}
\item 手机多媒体软件自动调试工具开发

\item 手机多媒体软件开发
\end{itemize}

\subsubsection{2004\thinspace 年\thinspace 10\thinspace 月 - 2005\thinspace 年\thinspace 9\thinspace 月}
\label{sec-1-0-7}
\textbf{Software Engineer},麒麟软件

\begin{itemize}
\item 企业集成应用软件测试
\end{itemize}

\section{自由软件项目}
\label{sec-2}

\subsubsection{Emacs}
\label{sec-2-0-1}

\begin{description}
\item[{\href{http://github.com/baohaojun/bbyac}{bbyac.el}}] Emacs\thinspace 下的补齐工具(Emacs-lisp)

\item[{\href{https://github.com/baohaojun/org-jira}{org-jira.el}}] Emacs\thinspace 下用\thinspace org-mode\thinspace 来进行\thinspace Jira\thinspace 开发流程管理的工具
(Emacs-lisp)

\item[{\href{https://github.com/baohaojun/ajoke}{Ajoke.el}}] 一个\thinspace Emacs\thinspace 下的\thinspace Java/Android\thinspace 集成开发环境
\end{description}

\subsubsection{Android}
\label{sec-2-0-2}
\begin{description}
\item[{\href{https://github.com/baohaojun/BTAndroidWebViewSelection}{CrossDict}}] Android\thinspace 下的英文字典软件,在 \href{https://play.google.com/store/apps/details?id=com.baohaojun.crossdict}{Google Play} 上发布(Java,Android)。用\thinspace Ajoke.el\thinspace 开发
\item[{\href{https://github.com/SmartisanTech/Wrench}{Wrench}}] 用PC连接、控制手机的工具软件,允许流畅同步显示手机屏幕,用Lua编程语言录制屏幕操作脚本,在PC端显示手机端通知消息等。用Qt + Lua开发,支持所有主流操作系统
\end{description}

\subsubsection{Input Method}
\label{sec-2-0-3}
\begin{description}
\item[{\href{https://github.com/baohaojun/system-config/tree/master/gcode/scim-cs/ime-py}{sdim}}] 跨所有主流平台(Win32/Linux/Mac OS\thinspace 甚至\thinspace Emacs)的输入法
(Python,C++,ObjC,Emacs-lisp)

\item[{\href{https://github.com/scim-im/scim-fcitx}{scim-fcitx}}] GNU/Linux\thinspace 下的输入法,基于\thinspace scim\thinspace 和\thinspace fcitx\thinspace 移植(C++,
GNU/Linux)
\end{description}

\subsubsection{System Software}
\label{sec-2-0-4}
\begin{description}
\item[{\href{https://github.com/baohaojun/beagrep}{beagrep}}] 结合搜索引擎的源代码\thinspace grep\thinspace 工具,0.23\thinspace 秒\thinspace grep\thinspace 两\thinspace G\thinspace 代码(C\#,
Perl)

\item[{\href{https://github.com/baohaojun/system-config}{system-config}}] 其他一些较小的脚本/程序,均放在 \href{https://github.com/baohaojun}{github} 上用\thinspace git\thinspace 管理
\end{description}


\section{技术技能}
\label{sec-3}

\subsubsection{编程语言 \& 库}
\label{sec-3-0-1}
\begin{description}
\item[{熟练}] Perl,Python,Bash,Emacs Lisp,C,C++,Java,Lua,Qt

\item[{用过}] ObjC,C\#,PHP,Ruby
\end{description}
\subsubsection{写作}
\label{sec-3-0-2}
\begin{description}
\item[{文本}] Org-mode,Emacs
\end{description}
\subsubsection{版本管理}
\label{sec-3-0-3}
Git \& Gerrit
\subsubsection{系统管理}
\label{sec-3-0-4}
基于\thinspace Debian\thinspace 的\thinspace Linux\thinspace 发行版系统管理、Bash\thinspace 脚本编程

\section{教育}
\label{sec-4}

\subsubsection{1997 - 2001}
\label{sec-4-0-1}
本科,控制理论与工程,浙江大学
\subsubsection{2001 - 2004}
\label{sec-4-0-2}
硕士,控制理论与工程,中科院自动化所

\section{个人信息}
\label{sec-5}
\subsubsection{出生日期}
\label{sec-5-0-1}
1980\thinspace 年\thinspace 3\thinspace 月\thinspace 10\thinspace 日
\subsubsection{手机}
\label{sec-5-0-2}
18610314439
\subsubsection{E-mail}
\label{sec-5-0-3}
\href{mailto:baohaojun@gmail.com}{baohaojun@gmail.com}
\subsubsection{网址}
\label{sec-5-0-4}
\begin{description}
\item[{博客}] \url{http://baohaojun.github.io}
\item[{代码}] \url{https://github.com/baohaojun}
\item[{System-config}] \url{https://github.com/baohaojun/system-config}
\item[{System-config\thinspace 使用视频}] \url{https://www.youtube.com/watch?v=qp2b3-Guej0}
\item[{Wrench}] \url{https://github.com/SmartisanTech/Wrench}
\item[{Wrench\thinspace 视频}] \url{https://m.youtube.com/watch?v=re_bECYY0rM}
\item[{Bbyac}] \url{http://github.com/baohaojun/bbyac}
\item[{Org-jira}] \url{https://github.com/baohaojun/org-jira}
\item[{Ajoke}] \url{https://github.com/baohaojun/ajoke}
\item[{Beagrep}] \url{https://github.com/baohaojun/beagrep}
\end{description}

% Emacs 25.1.91.1 (Org mode 8.2.10)
\end{CJK*}
\end{document}
